\section{Containerizing}

\subsection{Tree/Node interaction}

Trees are abstract data types. This makes trees easy to reason about.
Implementing a tree requires dealing with a number of subtle issues.
Since Trees are usually described in terms of the Nodes comprising
the Tree, some thought about differentiating Trees and Nodes is in order.

\subsubsection{Implicit structure}

Given some class, record, structure or whatever language feature which contains
data and supports function invocation, adding left and right pointers,
appropriate comparators, and insert and find methods suffices to provide
dictionary capability to the given language feature. However, trees require a
root, and managing this root increases the complexity whatever application
system requires the data sorted into the tree. Provision needs to be made for
the root changing, or even being deleted.

\subsubsection{Tree container}

Another way to implement trees is to create a tree container which implements
insert, search, etc. The implementation provided by the tree container could
implement the functionality itself, or call an implementation provided by
the node element.

\subsubsection{The BST Node}

By definition, each node in a binary search tree contains a left child and
a right child, either or both of which may be nil.

Some operations on binary search trees, such as {\tt delete}, require
the node's parent as well as its children. The parent to any node can also
be stored as a reference or pointer, or the parent can be determined at
runtime.

Storing parent pointers is easy. When inserting a node as a child, set that
node's parent pointer to the inserting node.

Determining parent pointers at runtime is not conceptually much more difficult:
keep track of the parent at each node by passing the current node to the
recursive calls. Two disadvantages of this include longer argument lists
for recursive calls, and either returning both node and parent when necessary.
For languages which do not allow multiple return values, one of the arguments
will have to capture the state change for access by the calling method.
In theory, this is straightforward. From an engineering standpoint, requiring
both a return value and an argument state change violates the principle of
least astonishment.

Passing both node and parent back through the argument list is a pretty good
case for adding the parent node directly to the Node data structure.


\subsection{C++ templating}

\subsection{C struct inclusion}

\paragraph{private headers}

\subsection{Ruby module inclusion}

It's not difficult to encode binary search tree capability into a Ruby module.
The module provides all the necessary instance binary search tree methods, and
raises errors when necessary overriding has not been implemented in the
including class.

The usual caveats about instance variables referenced in modules applies,
particularly that the including class must define the key for the tree.
