\documentclass{article}

\usepackage{amsmath}
\usepackage{amssymb}
\usepackage{fancybox}
\usepackage{tikz}
\usepackage{comment}


\usepackage{cite}

\usepackage{listings}
\usepackage{color}
\usepackage{hyperref}

\definecolor{mygreen}{rgb}{0,0.6,0}
\definecolor{mygray}{rgb}{0.5,0.5,0.5}
\definecolor{mymauve}{rgb}{0.58,0,0.82}

\lstset{ %
  backgroundcolor=\color{white},   % choose the background color; you must add \usepackage{color} or \usepackage{xcolor}
  basicstyle=\footnotesize,        % the size of the fonts that are used for the code
  breakatwhitespace=false,         % sets if automatic breaks should only happen at whitespace
  breaklines=true,                 % sets automatic line breaking
  captionpos=b,                    % sets the caption-position to bottom
  commentstyle=\color{mygreen},    % comment style
  % deletekeywords={...},            % if you want to delete keywords from the given language
  escapeinside={\%*}{*)},          % if you want to add LaTeX within your code
  extendedchars=true,              % lets you use non-ASCII characters; for 8-bits encodings only, does not work with UTF-8
  frame=single,                    % adds a frame around the code
  keepspaces=true,                 % keeps spaces in text, useful for keeping indentation of code (possibly needs columns=flexible)
  keywordstyle=\color{blue},       % keyword style
  language=Octave,                 % the language of the code
  % otherkeywords={*,...},           % if you want to add more keywords to the set
  numbers=left,                    % where to put the line-numbers; possible values are (none, left, right)
  numbersep=5pt,                   % how far the line-numbers are from the code
  numberstyle=\tiny\color{mygray}, % the style that is used for the line-numbers
  rulecolor=\color{black},         % if not set, the frame-color may be changed on line-breaks within not-black text (e.g. comments (green here))
  showspaces=false,                % show spaces everywhere adding particular underscores; it overrides 'showstringspaces'
  showstringspaces=false,          % underline spaces within strings only
  showtabs=false,                  % show tabs within strings adding particular underscores
  stepnumber=2,                    % the step between two line-numbers. If it's 1, each line will be numbered
  stringstyle=\color{mymauve},     % string literal style
  tabsize=2,                   % sets default   tabsize to 2 spaces
  title=\lstname                   % show the filename of files included with \lstinputlisting; also try caption instead of title
}

% \language=[Objective]{Caml}




\title{Some notes on hash trees}
\date{\today}
\author{David M. Doolin}


\begin{document}

\maketitle

\abstract{A few notes on hash trees and their implementations
in various programming languages.}

\tableofcontents

\section{Introduction}


The following topics will be discussed:

\begin{itemize}

\item Differences in implementations between programming languages.
\item How best to containerize, including pure tree node, tree wrapper with
node class, abstracting keys.
\item Theory and performance.

\end{itemize}


\section{Literature review}

Skiena~\cite[pp. 77, 370, 375, 589]{skiena} has a few notes.

Aho and Ullman~\cite[pp. 210]{rosen} define height and depth for nodes in a binary tree.

\begin{quote}
The height of a node n is the length of a longest path from n to
a leaf. The height of the tree is the height of the root. The depth or level of
a node n is the length of the path from the root to n.
\end{quote}

Cormen et al.~\cite{cormen:th:1990} remains a classic reference.

Rosen~\cite[pp. 757-760]{rosen} discusses three uses for binary search trees:

\begin{enumerate}
\item storing items from a list such that they can be easily found;
\item finding an object in a collection of similar objects;
\item efficiently encoding characters in a bit string.
\end{enumerate}

Sedgewick~\cite{sedgewick:r1990}.

O'Rourke~\cite{orourke:j1998}.

Manber~\cite[p. 87, Ex. 4.8]{manber:u1989} provides an interesting exercise:
show the AVL tree formed by inserting the ordered sequence $[1, 2, \ldots, 20]$.

\section{The Hash Tree}

Two main ideas:

\begin{enumerate}
\item Operations for manipulating BSTs;
\item intrinsic properties of BSTs.
\end{enumerate}

Operations include:

\begin{itemize}
\item insert node
\end{itemize}

Properties include:

\begin{itemize}
\item height
\end{itemize}


\subsection{Operating on the hash tree}

\subsection{Links}

\begin{itemize}
\item \href{http://aofa.cs.princeton.edu/60trees/}{http://aofa.cs.princeton.edu/60trees/}
\end{itemize}


% \section{References}

\bibliography{references}{}
\bibliographystyle{plain}


\appendix

\section{Implementation}


\end{document}
