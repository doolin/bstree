\section{The Binary Search Tree}

List of relevant theorems and corollaries:

\begin{enumerate}
  \item $n$ vertices implies $n-1$ edges (Rosen~\cite[p. 752]{rosen}).
  \item Number of nodes $n$ in a complete binary tree of height $h$ is $2^{h+1} - 1$
    (Aho and Ullman~\cite[p. 257]{aho:av:1992}).
  \item A full binary tree is with $i$ internal vertices contains $n = 2i + 1$ vertices.
    (Rosen~\cite[p. 752]{rosen}).
  \item Lookup time $T$ on a binary search tree is governed by the recurrence
    $T(h) \leq T(h-1) + O(1)$ (Aho and Ullman~\cite[p. 256]{aho:av:1992}).
  \item There are at most $2^h$ leaves in a binary tree of height $h$ (Rosen~\cite[p. 754]{rosen}).
  \item Search hits in BST from $N$ random keys require $\tilde2\ln N$ (about $1.39\ln N$) compares
    on average (Sedgewick and Wayne~\cite[p. 403]{sedgewick:r2011}).
  \item Sum of heights of complete binary tree, from Manber~\cite[p. 34]{manber:u1989}.
  \item AVL tree height from Manber~\cite[p. 75]{manber:u1989}.
  \item Complete binary tree embedding on a lattice, from Manber~\cite[p. 263]{manber:u1989}.
  \item Range searching with BST in Sedgewick~\cite[p. 373]{sedgewick:r1990}.
  \item V \& H line intersection with BST from Sedgewick~\cite[p. 391]{sedgewick:r1990}.
  \item Trapezoidal locations with BST, from O'Rourke~\cite[p. 289]{orourke:j1998}.
\end{enumerate}

The theorems need to get moved to properties. Some of the above list should be
moved to a future Applications section.

\subsection{Vertex-edge relation}

\textit{Theorem: A tree with $n$ vertices has $n-1$ edges.}

Basis step: $n = 1$: a tree $T$ with 1 vertex has 0 edges.

Inductive step:

\begin{enumerate}
  \item Inductive hypothesis: a tree $T$ with $k$ vertices has $k-1$
    edges.

  \item Advance the argument: Suppose $T$ with $k+1$ vertices, and $v$ a
    leaf of $T$ with parent $w$. We want to show $k$ edges.

  \item Remove $v$ and the edge $\overline{vw}$ to produce $T'$ with $k$
    vertices. $T'$ is connected with no circuits.
  \item TODO Apply Inductive hypothesis.
  \item TODO Complete inductive step.
\end{enumerate}
