\documentclass{article}

\usepackage{cite}

\title{Some notes on binary search trees}
\date{\today}
\author{David M. Doolin}


\begin{document}

\maketitle

\abstract{A few notes on binary search trees and their implementations
in various programming languages.}



\section{Introduction}


The following topics will be discussed:

\begin{itemize}

\item Differences in implementations between programming languages.
\item How best to containerize, including pure tree node, tree wrapper with
node class, abstracting keys.
\item Theory and performance.

\end{itemize}


\section{Literature review}

Skiena 2008~\cite{[pp. 77, 370, 375, 589]skiena} has a few notes.


Rosen (2012)~\cite[pp. 757-760]{rosen} discusses three uses for binary search trees:

\begin{enumerate}
\item storing items from a list such that they can be easily found;
\item finding an object in a collection of similar objects;
\item efficiently encoding characters in a bit string.
\end{enumerate}

\section{Containerizing}

\subsection{C++ templating}

\subsection{C struct inclusion}

\subsection{Ruby module inclusion}

Not sure if this will work, but it would be cool to have a Ruby module which
adds binary search tree capability to any object. The methods would have to be
class methods.

\section{References}

\bibliography{references}{}
\bibliographystyle{plain}

\section{Summary}

\appendix



\end{document}
