\section{Literature review}

Skiena~\cite[pp. 77, 370, 375, 589]{skiena} has a few notes.

Aho and Ullman~\cite[pp. 210]{aho:av:1992} define height and depth for nodes in a binary tree.

\begin{quote}
The height of a node n is the length of a longest path from n to
a leaf. The height of the tree is the height of the root. The depth or level of
a node n is the length of the path from the root to n.
\end{quote}

Cormen et al.~\cite{cormen:th:1990} remains a classic reference.

Rosen~\cite[pp. 757-760]{rosen} discusses three uses for binary search trees:

\begin{enumerate}
\item storing items from a list such that they can be easily found;
\item finding an object in a collection of similar objects;
\item efficiently encoding characters in a bit string.
\end{enumerate}

Sedgewick~\cite{sedgewick:r1990}.

O'Rourke~\cite{orourke:j1998}.

Manber~\cite[p. 87, Ex. 4.8]{manber:u1989} provides an interesting exercise:
show the AVL tree formed by inserting the ordered sequence $[1, 2, \ldots, 20]$.

F. Thomson Leighton discusses binary trees used in parallel computing~\cite[p. 407]{leighton:ft1992}.
