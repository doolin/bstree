\section{AVL tree}

AVL trees are a type of binary search tree which maintains height balance
on insert operations by \textit{retracing} the insertion path of the node,
and performing \textit{node rotation} at every node in the retrace path
which became unbalanced on the insert. AVL trees are not that technically
difficult to implement, but there are a few tedious details requiring
precise attention. Some factors to consider:

\begin{enumerate}
  \item Where the retracing is implemented depends on how the tree and
    node data structures are implemented. Because the root node may change
    as a result of rotation, the Tree data structure will have to have a way
    of ensuring the current root is valid.
  \item Retracing can be done from the tree, or as part of the node capability.
  \item Rotations ultimately decompose into two cases. The first case is moving
    a left (right) child into its parent's position, the other case is swapping a right (left)
    child for its parent, where both cases require ensuring all relevant pointers
    are updated accordingly.
\end{enumerate}

We'll work our way backwards through the above list, starting with a detailed
discussion of rotation, which is the fun part everyone wants to do anyway.

\subsection{Rotation}

There are plenty of references, web sites, Youtube videos and whatnot
where AVL rotation is explained in detail. None of them are wrong, but
most go into extraneous detail far too quickly, presenting situations
which will be hard to visualize during implementation, and possibly
difficult to construct for testing. It turns out, as claimed above,
that rotations are fundamentally about parent/child repositioning
while ensuring all the relevant pointers are updated accordingly.

Ironically, it's easier to motivate rotations using 3 nodes instead
of two nodes. The idea is putting a binary search tree back into
balance, and by definition, and AVL tree of size 2 is balanced.
So we need 3 nodes to see what the problem is, and how to solve it.

Three nodes results in 5 different binary search trees. We may discard
the balanced tree from our conversation immediately, leaving the other
4 trees out of balance, by definition. Of these 4 trees, we can use
symmetry to dispose of all the remaining trees with nodes right (or left)
leaving two cases to consider. While both cases are ``skew'' trees
(all the nodes have one and only one child), it's useful to note
that Case 1 has left children only, and Case 2 has a left child then
a right child. (Conversely, right children only, and right then
left child). We'll call Case 1 a \textit{left chain} and Case 2 a
\textit{left knee}.

FIGURES HERE.

\subsubsection{Left chain}

Suppose in the left chain we have a root keyed 17, and child left
7, then 2. For balance, we want the new root to be 7, with right 17
and left 2.

It's helpful here to distinguish between the root of the tree, and a
some local \textit{rotation root}. In this base case, the tree root
and the rotation root are the same node. In the general case, the
tree root and rotation root will be different, and we're going to
lean on our rapidly developing data structures intuition to keep the
context straight.

Continuing, the node keyed 17 is chosen as the \textit{rotation root}.
The node 7 we'll call the \textit{pivot}. Implementing this rotation
to balance the tree requires:

\begin{enumerate}
  \item Save the parent pointer of the rotation root (nil in this case).
  \item Move the node 17 to 7's right child, updating 17's parent pointer.
  \item Set 7's parent pointer to the saved value.
\end{enumerate}

\subsubsection{Left knee}

In the left knee, let's have our root 17, with left 7, and 7's right
child 11. In this case, we want 11 to be the root, with 7 as left
child and 17 as right child. Simply moving 7 into the tree root
position won't work, it will violate the binary search tree definition.
And besides, we can't put 17 on 7's right, 11 is already there.
Instead, what we do first is rotate to make a chain, then rotate
our new chain into balance.

And it turns out making a chain from knee is easy: choose the middle
node as the rotation root, the right child as the pivot, and rotate
counterclockwise. Now the tree looks like $7 \leftarrow 11\leftarrow 17$, and we
already know how to balance a left chain.

Above, we threw away the right side of the tree by invoking symmetry.
By invoking symmetry again, we get the right side of the tree back,
swapping `left' for `right' and `right' for `left.'


\paragraph{Why the base cases are important}
Invoking the property of the AVL tree, inserting a node triggers retracing,
which may result in rotations to rebalance.  Given random keys, it's
at least average behavior (I claim!) that an AVL tree is going to rebalance after
inserting the third node. The chain and knee situations above must be handled
correctly to proceed with general implementation of AVL tree.


\subsection{Rotation beyond the base cases}

The above is all well and good for trees of size 3, which is to say, useless.
Child nodes are subtrees with possibly very large subtrees of their own,
and inserting a node which travels all the way to a leaf at the bottom
of the tree might trigger rotational adjustment all the way back to the
root of tree. This is less of a problem than it would seem: the base
case implementation requires one additional concept, that of a
\textit{swing node} which changes its parent pointer from the pivot to
the pivot's previous child position on the rotation root. While that
seems complicated, and may not be exactly trivial, it's a relatively
straightforward extension of the base case rotation.

That is to say, our new friends chain and knee still control the structure
of the rotation.


\subsection{Retracing}

Retracing is the operation of iterating from the inserted
node up the tree, possibly (always?) to root, checking
the balance at each node and rebalancing if necessary.
Rebalancing is simply applying the correct rotations,
either the simple right or left, or the double rotations
as the case may be.

Retracing has three challenges:

\begin{enumerate}
  \item Determining or maintaining balance factor at each parent,
  \item distinguishing between ``chain'' and ``knee'' rotations, and
  \item Ensuring the relationship between the rotation root, the pivot
    and the rotation root's parent is correct.
\end{enumerate}

If the weight or ``balance factor'' is maintained at each node, it must be
adjusted during the rebalancing process. If balance factor isn't maintained at
each node, the height of each subtree at each node will be required, which is
expensive.

Some rotations will move the rotation root to a child position, which
requires resetting its parent to the pivot, and ensuring that the
pivot and the rotation root's previous parent are linked.

\paragraph{Chain versus knee rotations}

\subsection{Implementation design}

Assuming we want to maintain an accurate node pointer for the root of the tree,
at the highest level, retracing needs to be controlled by the same data
structure in charge of the root node. In a node/tree situation, retracing
belongs to the tree. Since retracing is performed bottom up from the
inserted node, the tree data structure needs to keep a reference to
the inserted node, which requires insertion to be controlled by the
tree as well.

One critical part for correctly implementing an AVL tree's knowledge of
it's height balance at every node. For example, the reference implementation
displayed on Wikipedia manages height balance by adjusting the
\textit{balance factor} on every rotation, and during retracing. That is,
balance factors are adjusted in two places. The reader is excused if
he or she finds splitting the work confusing, it is confusing. If possible,
balance factors should be updated in one place.
