\section{Properties of BST}

Properties of the data structure include:

\begin{enumerate}
  \item the keys are required to posess a \textit{total order}; every key must
        be comparable to every other key.
\end{enumerate}

The keys in a BST are generally assumed to be unique, hence little guidance on
implementation which may require handling keys which are not unique. Unique keys
can be handled in two ways:

\begin{enumerate}
  \item reject the node with the duplicate key with error (or whatever);
  \item replace the existing node with new node.
\end{enumerate}

For an implementation where nodes are externally created, then inserted,
the second case might allow for updating the value associated with the existing
node with the value of the incoming node containing the duplicate key.

Non-unique keys would need to be managed with some sort of data structure at
each node. Could be interesting to implement.

Properties include:

\begin{itemize}
  \item height
  \item depth
  \item maximum key
  \item minimum key
  \item size
  \item diameter
  \item class (full, perfect, complete, balanced, etc.)
\end{itemize}

Node properties include:

\begin{itemize}
  \item visited and unvisited
  \item has unvisited children
\end{itemize}

\subsection{Structural uniqueness of binary search tree}

Hypothesis: each different preorder traverse of a binary search
tree corresponds to a unique binary search tree. In other words,
every structurally unique binary search
tree has a unique preorder traverse.

I think this is true.

If it is true, it provides an easy to compare binary
search trees, just compare the result of an inorder
traverse, or define a function which compare nodes in
parallel during an inorder traverse of each.

Keys are assumed to be unique.

\subsection{Height}

\subsection{Depth}

\subsection{Size}

\subsection{Diameter}

\subsection{Classification}

Binary search trees may be classified according to the
arrangement of the leaves and internal nodes.
Binary search tree classifications include
full, perfect, complete, balanced, unbalanced, and degenerate.

First, how these classifications are defined in the literature is
worth some discussion, as different authors define trees in slightly
different ways. For example, Aho and Ullman define full and complete
trees as the same~\cite[p. 257]{aho:av:1992}, whereas Cormen
et al.~\cite[p. 1178]{cormen:th2009} distinguish between full and
complete binary trees.

The definitions which follow will be used consistently through this
work. The reader is free, of course, to take issue and use definitions
more to his or her preference.

\begin{itemize}
  \item Full: A rooted tree is called an m-ary tree if every internal
    vertex has no more than m children. The tree is called a full m-ary
    tree if every internal vertex has exactly m children. An m-ary tree
    with m = 2 is called a binary tree.~\cite[p. 748]{rosen} Also see
    Cormen et al.~\cite[p. 1178]{cormen:th2009}.

  \item Complete: A binary tree is complete when each child of the root node,
    left and right, is a complete subtree (Aho and Ullman). Cormen et al.\
    define complete as all leaves having the same depth with all nodes having
    having degree 2, that is, 2 children. This is equivalent to saying all
    internal nodes are full~\cite[p. 1179]{cormen:th2009}).
\end{itemize}

\subsection{Maximum and minimum keys}

\setcounter{sno}{0}

\sno Maximum and minimum elements of binary search trees are probably the easiest
algorithms to both understand and implement. \sno The maximum is the right-most
node, the minimum the left-most, finding either means traversing the tree
to the appropriate leaf, and returning that node which has no children.

\sno The maximum function written in Ruby is particularly elegant,
we take advantage of Ruby's safe navigation operator {\tt \&}:

\lstset{language={Ruby}}
\begin{lstlisting}[frame=none]
def maximum
  right&.maximum || self
end
\end{lstlisting}


\section{Building a binary search tree}

\subsection{Collisions or duplicate values}


None the preceding references explicitly discuss collisions.

\begin{itemize}
\item The duplicate key can be added to the tree, either silently or with a warning.
\item The duplicate key is not added to the tree, either silently or with a warning.

\item In Ruby, does replacing a node in a tree leak memory, or will the deleted node
get garbage collected? How about Python?

\item How to delete in c/c++ such that memory isn't leaked? This should be easier,
call delete after the node is orphaned.
\end{itemize}
